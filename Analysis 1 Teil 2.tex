\documentclass[headsepline,12pt,a4paper]{scrartcl}
\usepackage{amsmath, amssymb}
\linespread{1.3}
\usepackage[ngerman]{babel}
\usepackage[T1]{fontenc} 
\usepackage{times}
\setlength{\parindent}{0cm}
\usepackage[utf8]{inputenc}
\usepackage{physics}

\makeatletter
\def\myItem{%
   \@ifnextchar[ \@myItem{\@noitemargtrue\@myItem[\@itemlabel]}}
\def\@myItem[#1]{\item[#1]\mbox{}\\}
\makeatother % Wir definieren uns hier eine sauberere myItem Funktion als \item[]

\makeatletter
\def\myItemc{%
   \@ifnextchar[ \@myItem{\@noitemargtrue\@myItem[\@itemlabel]}}
\def\@myItemc[#1]{\item[#1]\mbox{}\\}
\makeatother % My Item, aber automatisch gecentert // funktioniert noch nicht so wirklich

\begin{document}

\begin{titlepage}
	\centering
	{\huge\bfseries Analysis 1\par}
	\vspace{2cm}
	{\Large\itshape Noah J. Zimmermann\par}
	\vfill
	Teil\par
	 \textsc{II}

	\vfill

% Bottom of the page
	{\large Wintersemester 2016/17\par}
\end{titlepage}

\tableofcontents

\newpage

\section*{Raum reelwertiger stetiger Funktionen}
\addcontentsline{toc}{section}{Raum reelwertiger stetiger Funktionen}

\begin{itemize}

\myItem[Raum reelwertiger stetiger Funktionen]

$$ C(\mathbb{K}):=\{f:\mathbb{K} \rightarrow \mathbb{R} | f \textit{ ist stetig auf } \mathbb{K}\} $$

ist der Raum reelwertiger stetiger Funktionen auf $\mathbb{K}$ \\

Bemerkung: \\

Seien $f,g \in C(\mathbb{K}), \lambda \in \mathbb{R}$. Dann ist auch $f+g$, $f\cdot g$, $\lambda f$ wieder eine Funktion aus $C(\mathbb{K}). C(\mathbb{K})$ bildet dann einen Ring. \\

\myItem[Definition] Seien $f,g:\mathbb{K} \rightarrow \mathbb{R}$ \\

$\max\limits_{x\in \mathbb{K}}(f,g)(x):= \max\limits_{x\in \mathbb{K}}(f(x),g(x))$ \\

Analog für das Minimum. Dies ist lediglich Notation. \\

\myItem[Satz]

$\max (f,g)$ und $\min (f,g)$ sind in $C(\mathbb{K})$ für $f,g \in C(\mathbb{K}$ \\

Man stelle sich einfach vor, dass man eine $|f|$ als Komposition mit $f$ darstellen kann und dann ist $|f|$ wieder stetig. \\

Außerdem gilt: $min(f,g)= -max(-f,-g)$ \\

und dann definiert man $$ |f|_{\infty} := \max|f(x)| $$\\

\section*{Norm}
\addcontentsline{toc}{section}{Norm}

\myItem[Norm]

Sei $\mathbb{K}$ ein Körper (mit dem Betrag ||), sei $V$ ein VR über $\mathbb{K}$. \\

$\| \cdot \|: V \rightarrow \mathbb{R}$

heißt eine Norm auf $V$ genau dann wenn: \\

\begin{enumerate}

\item Definitheit: \\

$\forall x \in V  : \| x \| \geq 0 \wedge (\| x \| = 0 \Leftrightarrow x = 0) $\\

\item Homogenität: \\

$\forall x \in V: \alpha \in \mathbb{K} \; \; \| \alpha x \| = |\alpha | = \| x \|$ \\

\item Dreiecksungleichung: \\

$\forall x,y \in V: \| x+y \| \leq \| x \| + \| y \|$ \\


\end{enumerate}

$(V, \| \|)$ heißt normierter Vektorraum. \\

\newpage

\myItem[Beispiel für normierte Vektorräume]
$C([a,b])$ nach $\mathbb{R}_0^{+}$ ist demnach ein Vektorraum. ( Einfach nachrechnen )  \\

Man definiert eine \textbf{Normkonvergenz}

$$ f_n \xrightarrow[n \rightarrow \infty] f $$ in Norm das gleiche wie $$\| f- f_n\| \xrightarrow[n \rightarrow \infty] 0 $$

Demnach ist die Konvergenz ist Norm mit der gleichmäßigen Konvergenz gleichzusetzen, weil die Funktionen ja praktisch keinen Unterschied mehr haben, da sie sich 0 annähern und da es von $x$ unabhängig ist, somit ist es nicht nur punktweise ( von $x$ abhängig ) sondern gleichmäßig ( von $x $ unabhängig ) konvergent. \\

\myItem[Lemma]

Somit gilt:

Für eine Funktionenfolge $(f_n)_{n \in \mathbb{N}} \in C([a,b])$ ist die gleichmäßige Konvergenz gegen eine Grenzfunktion $f: [a,b] \rightarrow \mathbb{R}$ gleichbedeutend mit $\| f- f_n\| \xrightarrow[n \rightarrow \infty] 0$

\myItem[Cauchy Konvergenz von Funktionenfolgen]

Man kann die Cauchykonvergenz auch auf Funktionenfolgen anwenden: \\
Eine Funktionenfolge $(f_n)_{n \in \mathbb{N}} \in C([a,b])$ heißt Cauchy Folge, wenn

$$ \forall \varepsilon \; \exists n_\varepsilon \in \mathbb{N} \; : n,m \geq n_\varepsilon 
\Rightarrow \| f_m- f_n\| < \varepsilon $$

Dann ist jede konvergente Folge wieder eine Cauchy Folge. Der Beweis folgt analog zu den Zahlenfolgen. \\




\end{itemize}

\end{document}


